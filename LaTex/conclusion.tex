\documentclass[Deriaz_Traiber_Labo02]{subfiles}


\begin{document}
\chapter{Conclusion générale}
Lors de ce laboratoire, nous avons pu réaliser les 2 antennes (dipôle et patch), chacune sur 2 substrats différents (FR-4 et céramique). Nous avons observé que le réglage de la fréquence de résonance peut être réalisé avec plusieurs paramètres et il rapide d'obtenir la bonne fréquence. Dans le cas de l'antenne patch, la bande passante est difficile à régler.\\
Chaque antenne a été réalisée avec une méthode différente
\begin{itemize}
\item Antenne dipôle : Ajustage des paramètres en utilisant le "sweep" de CST afin de simuler plusieurs scénarios et de choisir le meilleur.
\item Antenne patch : Test de quelques valeurs puis interpolation entre elles pour déterminer la meilleure.
\end{itemize}
Les deux approches se sont montrées efficaces. La méthode sweep est plus précise et donne un meilleur aperçu du comportement de l'antenne face à un changement de paramètre. La méthode interpolation, en revanche, est plus rapide.\\
Les antennes dipôles réalisées respectent le cahier des charges quant aux antennes patch, elles respectent la fréquence de résonance mais ne possède pas une bande passante suffisante.
\end{document}